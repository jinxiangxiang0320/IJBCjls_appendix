\documentclass{ws-ijbc}
\usepackage{ws-rotating}     % used only when sideways tables/figures are used
\usepackage{graphicx}
\usepackage{epstopdf}
\usepackage{adjustbox}
\usepackage{amsmath,amssymb}
\usepackage{booktabs}
\usepackage{graphicx}
\usepackage{subfigure}
\usepackage{natbib}
\usepackage{listings}
\usepackage{xcolor}
\setcitestyle{numbers,square}
\newtheorem{thm}{Theorem}[section]
\newtheorem{rem}{Remark}
\newtheorem{cor}{Corollary}
\newtheorem{con}{Conjecture}


\begin{document}

\catchline{}{}{}{}{} % Publisher's Area please ignore


\markboth{X.Jin et al.}{Coexistence and Bifurcations in a Tri-Trophic Intraguild Predation Model}

\title{Appendixes are included for review only, which will be removed if the paper can be accepted for publication.
}

\author{Xiangxiang Jin}

\address{School of Sciences, Southwest Petroleum University\\
 Chengdu, Sichuan 610500, China}

\author{Lingling Liu}
\address{School of Sciences, Southwest Petroleum University\\
 Chengdu, Sichuan 610500, China\\
Data Recovery Key Laboratory of Sichuan Province\\
College of Mathematics and Information Science, Neijiang Normal University\\
Neijiang, Sichuan 641100, China\\
a600aa@163.com}

\author{Yong-Guo Shi\footnote{Corresponding author.
E-mail addresses: scumat@163.com}}
\address{Data Recovery Key Laboratory of Sichuan Province\\
College of Mathematics and Information Science, Neijiang Normal University\\
Neijiang, Sichuan 641100, China\\
scumat@163.com}



\maketitle
\newpage


\section*{Appendixes are included for review only, which will be removed if the paper can be accepted for publication.}

\appendix{Sets and matlab code of $\mathcal{S}$ in Theorem 2.2}
\subsection*{A.1~Sets of parameters}
\begin{equation*}
  \begin{aligned}
  \mathcal{I}_{3}\cap\mathcal{B}_{2}&=\left\{\mu\in\mathcal{R}
  \big| 0<d_{1}<\frac{1}{1+\alpha_{1}}, e_{2}\varrho<\varpi, \frac{e_{2}}{h}<d_{2}<\frac{1}{1+\alpha_{2}}+\frac{e_{2}}{h}, \frac{d_{1}}{1-\alpha_{1}d_{1}}<\kappa_{1}<\kappa_{2}<1 \right\},\\
  \mathcal{I}_{3}\cap\mathcal{B}_{11} &= \left\{\mu\in\mathcal{R} \big|
  0<d_{1}<\frac{1}{1+\alpha_{1}}, e_{2}\varrho<\varpi, \max\{\frac{1}{1+\alpha_{2}},\frac{e_{2}}{h}\}<d_{2}<\min\{\frac{1}{1+\alpha_{2}}+\frac{e_{2}}{h},1\}, \right.\\
  &\qquad\left.\frac{d_{1}}{1-\alpha_{1}d_{1}}<\kappa_{1}<\kappa_{2}<1 \right\},\\
    \mathcal{I}_{21}\cap\mathcal{B}_{2}&=\left\{\mu\in\mathcal{R}
  \big| 0<d_{1}<\frac{1}{1+\alpha_{1}}, e_{2}\varrho<\varpi, \frac{e_{2}}{h}<d_{2}<\frac{1}{1+\alpha_{2}},\kappa_{2}<1 \right\},\\
  \mathcal{I}_{22}\cap\mathcal{B}_{2}&=\left\{\mu\in \mathcal{R}
\big|0<d_{1}<\frac{1}{1+\alpha_{1}},e_{2}\varrho\geq\varpi, \frac{e_{2}}{h}<d_{2}<\frac{1}{1+\alpha_{2}},\frac{d_{1}}{1-\alpha_{1}d_{1}}<\kappa_{2}<1 \right\},\\
\mathcal{I}_{21}\cap\mathcal{B}_{11}&=\left\{\mu\in \mathcal{R}
\big|0<d_{1}<\frac{1}{1+\alpha_{1}},e_{2}\varrho<\varpi, \max\{\frac{1}{1+\alpha_{2}},\frac{e_{2}}{h}\}<d_{2}<\min\{\frac{1}{1+\alpha_{2}}+\frac{e_{2}}{h},1\}, \kappa_{2}<1 \right\},\\
\mathcal{I}_{22}\cap\mathcal{B}_{11}&=\left\{\mu\in \mathcal{R}
\big|0<d_{1}<\frac{1}{1+\alpha_{1}},e_{2}\varrho\geq\varpi, \max\{\frac{1}{1+\alpha_{2}},\frac{e_{2}}{h}\}<d_{2}<\min\{\frac{1}{1+\alpha_{2}}+\frac{e_{2}}{h},1\},\right.\\
&\qquad\left.\frac{d_{1}}{1-\alpha_{1}d_{1}}<\kappa_{2}<1 \right\},\\
\mathcal{I}_{11}\cap\mathcal{B}_{2}&=\left\{\mu\in \mathcal{R}
\big|0<d_{1}<\frac{1}{1+\alpha_{1}},e_{2}\varrho<\varpi,\frac{e_{2}}{h}<d_{2}<\min\{\frac{1}{1+\alpha_{2}},\kappa_{2}<1 \right\},\\
\mathcal{I}_{12}\cap\mathcal{B}_{2}&=\left\{\mu\in \mathcal{R}
\big|0<d_{1}<\frac{1}{1+\alpha_{1}},e_{2}\varrho\leq\varpi,\frac{e_{2}}{h}<d_{2}<\min\{\frac{1}{1+\alpha_{2}},\kappa_{1}<\frac{d_{1}}{1-\alpha_{1}d_{1}}, \kappa_{2}<1 \right\},\\
\mathcal{I}_{13}\cap\mathcal{B}_{2}&=\left\{\mu\in \mathcal{R}
\big|0<d_{1}<\frac{1}{1+\alpha_{1}},e_{2}\varrho<\varpi,\frac{e_{2}}{h}<d_{2}<\min\{\frac{1}{1+\alpha_{2}},\frac{d_{1}}{1-\alpha_{1}d_{1}}<\kappa_{1}<1<\kappa_{2} \right\},\\
\mathcal{I}_{14}\cap\mathcal{B}_{2}&=\left\{\mu\in \mathcal{R}
\big|0<d_{1}<\frac{1}{1+\alpha_{1}},e_{2}\varrho<\varpi,\frac{e_{2}}{h}<d_{2}<\min\{\frac{1}{1+\alpha_{2}},\kappa_{1}<\frac{d_{1}}{1-\alpha_{1}d_{1}}<\kappa_{2}<1 \right\},\\
\mathcal{I}_{01}\cap\mathcal{B}_{2}&=\left\{\mu\in \mathcal{R}
\big|0<d_{1}<\frac{1}{1+\alpha_{1}},e_{2}\varrho\geq\varpi,\frac{e_{1}+[(1+\alpha_{1})d_{1}-1]e_{2}}{e_{1}(1+\alpha_{2})}<d_{2}<\frac{1}{1+\alpha_{2}},  \kappa_{2}<\frac{d_{1}}{1-\alpha_{1}d_{1}}\right\},\\
\mathcal{I}_{02}\cap\mathcal{B}_{2}&=\left\{\mu\in \mathcal{R}
\big|0<d_{1}<\frac{1}{1+\alpha_{1}},e_{2}\varrho<\varpi, 0<d_{2}\leq\frac{e_{1}+[(1+\alpha_{1})d_{1}-1]e_{2}}{e_{1}(1+\alpha_{2})}, \kappa_{2}<\frac{d_{1}}{1-\alpha_{1}d_{1}} \right\},\\
\mathcal{I}_{03}\cap\mathcal{B}_{2}&=\left\{\mu\in \mathcal{R}\big|0<d_{1}<\frac{1}{1+\alpha_{1}},0<d_{2}<\frac{1}{1+\alpha_{2}},\kappa_{1}>1\right\},\\
\mathcal{I}_{04}\cap\mathcal{B}_{2}&=\left\{\mu\in \mathcal{R}\big|0<d_{1}<\frac{1}{1+\alpha_{1}},0<d_{2}<\frac{1}{1+\alpha_{2}},e_{2}\varrho>\varpi\right\},\\
\mathcal{I}_{11}\cap\mathcal{B}_{11}&=\left\{\mu\in \mathcal{R}
\big|0<d_{1}<\frac{1}{1+\alpha_{1}},e_{2}\varrho<\varpi, \max\{\frac{1}{1+\alpha_{2}},\frac{e_{2}}{h}\}<d_{2}<\min\{\frac{1}{1+\alpha_{2}}+\frac{e_{2}}{h},1\},
,\kappa_{2}<1\right\},\\
\mathcal{I}_{12}\cap\mathcal{B}_{11}&=\left\{\mu\in \mathcal{R}
\big|0<d_{1}<\frac{1}{1+\alpha_{1}},e_{2}\varrho\leq\varpi, \max\{\frac{1}{1+\alpha_{2}},\frac{e_{2}}{h}\}<d_{2}<\min\{\frac{1}{1+\alpha_{2}}+\frac{e_{2}}{h},1\},\right.\\
&\qquad\left.\kappa_{1}<\frac{d_{1}}{1-\alpha_{1}d_{1}}, \kappa_{2}<1\right\},\\
\end{aligned}
\end{equation*}
\begin{equation*}
\begin{aligned}
\mathcal{I}_{13}\cap\mathcal{B}_{11}&=\left\{\mu\in \mathcal{R}
\big|0<d_{1}<\frac{1}{1+\alpha_{1}},e_{2}\varrho<\varpi,\max\{\frac{1}{1+\alpha_{2}},\frac{e_{2}}{h}\}<d_{2}<\min\{\frac{1}{1+\alpha_{2}}+\frac{e_{2}}{h},1\},\right.\\
&\qquad\left.\frac{d_{1}}{1-\alpha_{1}d_{1}}<\kappa_{1}<1<\kappa_{2}\right\},\\
\mathcal{I}_{14}\cap\mathcal{B}_{11}&=\left\{\mu\in \mathcal{R}
\big|0<d_{1}<\frac{1}{1+\alpha_{1}},e_{2}\varrho<\varpi, \max\{\frac{1}{1+\alpha_{2}},\frac{e_{2}}{h}\}<d_{2}<\min\{\frac{1}{1+\alpha_{2}}+\frac{e_{2}}{h},1\},\right.\\
&\qquad\left.\kappa_{1}<\frac{d_{1}}{1-\alpha_{1}d_{1}}<\kappa_{2}<1\right\},\\
\mathcal{I}_{01}\cap\mathcal{B}_{11}&=\left\{\mu\in \mathcal{R}
\big|0<d_{1}<\frac{1}{1+\alpha_{1}},e_{2}\varrho\geq\varpi,\frac{1}{1+\alpha_{2}}\leq d_{2}<1, \kappa_{2}<\frac{d_{1}}{1-\alpha_{1}d_{1}}\right\},\\
\mathcal{I}_{03}\cap\mathcal{B}_{11}&=\left\{\mu\in \mathcal{R}\big|0<d_{1}<\frac{1}{1+\alpha_{1}},\frac{1}{1+\alpha_{2}}\leq d_{2}<1,\kappa_{1}>1\right\},\\
\mathcal{I}_{04}\cap\mathcal{B}_{11}&=\left\{\mu\in \mathcal{R}\big|0<d_{1}<\frac{1}{1+\alpha_{1}},\frac{1}{1+\alpha_{2}}\leq d_{2}<1,e_{2}\varrho>\varpi\right\}.
  \end{aligned}
\end{equation*}
\subsection*{A.2~Matlab code for $\mathcal{S}$}

\begin{lstlisting}[language=Matlab, caption={Matlab code for computing symbolic sequence $\mathcal{S}$}, label={lst:matlab_code}]
alpha = zeros(1,2); d = zeros(1,2); e = zeros(1,2); A = zeros(1,4);
alpha(1)=input('Please enter alpha[1]:'); alpha(2)=input('Please enter alpha[2]:');
d(1)=input('Please enter d[1]:'); d(2)=input('Please enter d[2]:');
e(1)=input('Please enter e[1]:'); e(2)=input('Please enter e[2]:');
h = input('Please enter h:');A(1)=e(1)*d(2)*(1+h) - e(2)*(d(1)+e(1));
A(2)=e(2)(1-alpha(1)*d(1))e(1)((alpha(2)d(2)-1-h)+(1-alpha(1)-alpha(2))(e(2)-h*d(2)));
A(3)=e(1)*alpha(1)*(e(2)-h*d(2)) + e(1)*(1-alpha(1))*(h + (e(2)-h*d(2))*alpha(2));
A(4)=e(1)*alpha(1)*(h + (e(2)-h*d(2))*alpha(2));
Delta=(A(2)*A(3)-9*A(1)*A(4))^2 - 4*(A(3)^2-3*A(2)*A(4))*(A(2)^2-3*A(1)*A(3));
symbol_Sequence = arrayfun(@(x) '+', A, 'UniformOutput', false);
symbol_Sequence(A < 0) = {'-'};
symbol_Sequence(A == 0) = {'0'};
if Delta > 0
    symbol_Sequence{5} = '+';
elseif Delta < 0
    symbol_Sequence{5} = '-';
else
    symbol_Sequence{5} = '0';
end
disp('symbolic Sequence S is:');
disp(['[A[0], A[1], A[2], A[3], Delta]=', ...
      '[', strjoin(symbol_Sequence, ','), ']']);
\end{lstlisting}

\subsection*{A.3~Expressions}

The expressions in Theorem~4.2 is
{\small
\begin{eqnarray*}
&&T_{1}\Psi_{1} +\Psi_{2} T_{2}=
\{-\alpha_{2}\gamma_{2}\left(\alpha_{2}-1\right)^{2}[\gamma_{1} \gamma_{2} \left(\alpha_{2}+4\right) \left(1+\alpha_{2}\right)^{3} \left(\alpha_{1} \alpha_{2}-\alpha_{1}+2 \alpha_{2}\right)^{2} e_{1}
+4\left( \alpha_{1}^{2}+4\alpha_{1}+2 \gamma_{1}+4\right) \alpha_{2}^{6}
\\&&~~~~~
+4\left(\alpha_{1} \gamma_{1} \gamma_{2}+2 \gamma_{1} \gamma_{2}+4 \alpha_{1}+8 \gamma_{1}+8\right) \alpha_{2}^{5}+\left(\left(-4 \gamma_{1} \gamma_{2}-8 \gamma_{1}-8\right) \alpha_{1}^{2}+\left(-4 \gamma_{1} \gamma_{2}-32 \gamma_{1}-16\right) \alpha_{1}+16
+\left(8 \gamma_{2}+24\right) \gamma_{1}\right)
\\&&~~~~~
\alpha_{2}^{4}
+\left(-4 \alpha_{1}^{2} \gamma_{1} \gamma_{2}+\left(-52 \gamma_{1} \gamma_{2}-32 \gamma_{1}-16\right) \alpha_{1}-72 \gamma_{1} \gamma_{2}\right) \alpha_{2}^{3}-20 \gamma_{2} \left(3\alpha_{1}+4\right) \gamma_{1} \alpha_{1} \alpha_{2}
+(\left(44 \gamma_{1} \gamma_{2}+8 \gamma_{1}+4\right) \alpha_{1}^{2}
\\&&~~~~~
+132 \alpha_{1} \gamma_{1} \gamma_{2}+56 \gamma_{1} \gamma_{2}) \alpha_{2}^{2}+24 \alpha_{1}^{2} \gamma_{1} \gamma_{2}]\};
\end{eqnarray*}
\begin{eqnarray*}
&&
P_{2}\mid\tilde{d}_{1}= -\gamma_{2}\vartheta_{1}e_{2}^{3} \left(\bar x \alpha_{1}+1\right)^{2} \left(\bar x \alpha_{2}+1\right) \left[\left(\bar x \alpha_{2}+1\right)\vartheta_{1}\vartheta_{2}\gamma_{1}+\bar x \left(\bar x -1\right) \left(\bar x \alpha_{1}+1\right)\right]+\vartheta_{1}\vartheta_{2}\{\gamma_{2} \left(\bar x \alpha_{1}+1\right)^{2}
\\&&~~~~~
\left[2 h \left(\bar x \alpha_{2}+1\right)\vartheta_{1}\vartheta_{2}\gamma_{1}+\bar x \left(h \bar x^{2} \alpha_{1}-h \bar x \alpha_{1}+h \bar x -h -1\right)\right]
+\bar x \gamma_{1} \left(\bar x \alpha_{2}+1\right)[\left(\bar x \alpha_{1}+1\right)^{2} \left(2 \bar x \alpha_{2}-\alpha_{2}+1\right) \vartheta_{1} h
\\&&~~~~~
+\bar x \alpha_{2}+1]\}-h \gamma_{1} e_{2}\vartheta_{1}^{2} \vartheta_{2}^{2} \{h \left(\bar x \alpha_{1}+1\right)^{2}\left[\vartheta_{2}\gamma_{2}+\bar x \left(2\bar  x \alpha_{2}-\alpha_{2}+1\right)\right]+\bar x \left(\alpha_{1}-\alpha_{2}\right)\};
\\&&
P_{1}'P_{2}\mid\tilde{d}_{1}=\{-\gamma_{2}\vartheta_{1}e_{2}^{3} \left(\bar x \alpha_{1}+1\right)^{2} \left(\alpha_{2} \bar x +1\right) \left[\left(\alpha_{2} \bar x +1\right)\vartheta_{1}\vartheta_{2}\gamma_{1}+\bar x \left(\bar x -1\right) \left(\bar x \alpha_{1}+1\right)\right]
+\vartheta_{1}\vartheta_{2}\{\gamma_{2} \left(\bar x \alpha_{1}+1\right)^{2}
\\&&~~~~~
\left[2 h \left(\alpha_{2}\bar  x +1\right)\vartheta_{1}\vartheta_{2}\gamma_{1}+\bar x \left(h\bar x^{2} \alpha_{1}-h\bar x \alpha_{1}+h \bar x -h -1\right)\right]
+\bar x \gamma_{1} \left(\alpha_{2} \bar x +1\right)[\left(\bar x \alpha_{1}+1\right)^{2} \left(2 \alpha_{2} \bar x
-\alpha_{2}+1\right)\vartheta_{1}h
\\&&~~~~~
+\alpha_{2}\bar  x +1]\}-h \gamma_{1} e_{2} \vartheta_{1}^{2}\vartheta_{2}^{2} \{h \left(\bar x \alpha_{1}+1\right)^{2} \left[\vartheta_{2}\gamma_{2}+\bar x \left(2 \alpha_{2} \bar x -\alpha_{2}+1\right)\right]+\bar x \left(\alpha_{1}-\alpha_{2}\right)\}
\}[\gamma_{1} e_{1} \vartheta_{2}^{2} h^{2}-\gamma_{1} e_{1} \left(\bar x \alpha_{2}+1\right) e_{2} \vartheta_{2} h
\\&&~~~~~
-\bar x \alpha_{2} e_{2}^{2}]
/[h \vartheta_{2}-e_{2} (\bar x \alpha_{2}+1)](\bar x \alpha_{2}+1)e_{1}e_{2}.
  \end{eqnarray*}
}

\appendix{Proof of Theorem 4.2.(ii-iii)}

\begin{proof}
$\mathbf{(ii)}$~~First, we compute the Jacobian matrix of the system at $E_{2}$ as follows
\begin{equation}\label{jaco5}
    J_{E_{2}}=\left(
   \begin{array}{ccc}
       d_{2}(1+\alpha_{2})-\frac{2d_{2}}{1-\alpha_{2}d_{2}} & \frac{d_{2}}{1+(\alpha_{1}-\alpha_{2})d_{2}} & -d_{2} \\
      0 & a_{22} & 0 \\
        \gamma_{2}\left[1-d_{2}(1+\alpha_{2})\right]& \frac{\gamma_{2}e_{2}[1-d_{2}(1+\alpha_{2})]}{(1-\alpha_{2}d_{2})^{2}} & 0 \\
     \end{array}
   \right),
\end{equation}
and then obtain the corresponding characteristic equation (\ref{eq48})
\begin{equation}\label{eq48}
(\lambda-\theta_{1})(\lambda^{2}+\theta_{2}\lambda+\theta_{3}d_{2})=0,
\end{equation}
where
\begin{equation*}
\begin{aligned}
&\theta_{1}=\frac{2d_{2}}{1-\alpha_{2}d_{2}}-d_{2}(1+\alpha_{2}),\quad \theta_{3}=\gamma_{2}\left[1-d_{2}(1+\alpha_{2})\right],\\
&\theta_{2}=\gamma_{1}\left[\frac{d_{2}}{1+(\alpha_{1}-\alpha_{2})d_{2}}-\frac{e_{1}(1-d_{2}(1+\alpha_{2}))}{(1-\alpha_{2}d_{2})^{2}}-d_{1}\right].
\end{aligned}
\end{equation*}
Assuming that equation (\ref{eq48}) has a pair of conjugate complex roots $\lambda_{1,2}=u_{2}(d_{2})\pm iv_{2}(d_{2})$ and $u_{2}(d_{2}^{0})=0$, then by substituting into the equation (\ref{eq48}) and taking the derivative for the bifurcation parameter $d_{2}^{0}=\frac{\alpha_{2}-1}{\alpha_{2}(1+\alpha_{2})}$, and separate the real and imaginary parts. Then we obtain that $
v_{2}(d_{2}^{0})=\frac{\sqrt{\gamma_{2}(\alpha_{2}^{2}-1)}}{\alpha_{2} \left(\alpha_{2} +1\right)}\neq0$.
The verification of the transversality condition for the Hopf bifurcation follows a process similar to the proof of Theorem 4.2, and thus is not detailed here for brevity. By calculation, we derive the following formula
\begin{equation}
 \frac{\partial u_{2}}{\partial d_{2}}|_{ d_2=d_{2}^{0}}=\frac{\Psi_{1}(d_{2})T_{1}(d_{2})+\Psi_{2}(d_{2})T_{2}(d_{2})}{T_{1}^{2}(d_{2})+T_{2}^{2}(d_{2})},
\end{equation}
where
\begin{equation*}
\begin{aligned}
&T_{1}(d_{2})=\theta_{1}\theta_{2}+4d_{2}\theta_{3},\quad \Psi_{2}(d_{2})=-v_{2}(d_{2})\left(\frac{\partial (\theta_{1}\theta_{2})}{\partial d_{2}}+\frac{\partial (d_{2}\theta_{3})}{\partial d_{2}}\right),\\
&\Psi_{1}(d_{2})=-\frac{\partial (\theta_{1}+\theta_{2})}{\partial d_{2}}+\frac{\partial (d_{2}\theta_{2}\theta_{3})}{\partial d_{2}},\quad T_{2}(d_{2})=2v_{2}(d_{2})(\theta_{1}+\theta_{2}).
\end{aligned}
\end{equation*}
Due to the extensive computation required, we used symbolic computation software. The transversality condition $\frac{\partial u_{2}}{\partial d_{2}}|_{ d_2=d_{2}^{0}}\neq0$ for a Hopf bifurcation is satisfied when the condition $T_{1}\Psi_{1} +\Psi_{2} T_{2}|_{ d_2=d_{2}^{0}}\neq0$ holds. Hence, system (2) will undergo a Hopf bifurcation at the boundary equilibrium $E_{2}$ when the parameter $d_{2}$ crosses the critical value $d_{2}^{0}$.\\
$\mathbf{(iii)}$~~
According to the characteristic equation (8) of the system at the interior equilibrium $\bar E$, the equation (8) can be written as 
\begin{equation}\label{eq411}
H(\lambda)=\lambda^{3}+P_{1}\lambda^{2}+P_{2}\lambda+P_{3}=(\lambda^{2}+P_{2})(\lambda+P_{1})=0.
\end{equation}
Thus, equation (\ref{eq411}) has a pair of pure imaginary characteristic roots and a negative root at $d_{1}=\tilde{d}_{1}$. Specifically, the roots are: $\lambda_{1}=\sqrt{P_{2}}i,\lambda_{2}=-\sqrt{P_{2}}i,\lambda_{3}=-P_{1}$. Next, we will verify the transversality condition for the Hopf bifurcation in system (2). First, we compute the partial derivative of the function (\ref{eq411}) with respect to $d_{1}$ as follows,
\begin{equation}\label{eq413}
\frac{\partial H(\lambda)}{\partial d_{1}}=(3\lambda^{2}+2P_{1}\lambda+P_{2})\lambda'+P_{1}'\lambda^{2}+P_{2}'\lambda+P_{3}'=0,
\end{equation}
further, we have
\begin{equation*}
\lambda'|_{\lambda=\lambda_{1,2}=\pm\sqrt{P_{2}}i}=\frac{\left(P_{2} P_{1}'\pm i P_{2}' \sqrt{P_{2}}+P_{3}'\right)\left(P_{2}\pm iP_{1} \sqrt{P_{2}}\right)}{2 P_{2} \left(P_{1}^{2}+P_{2}\right)},
\end{equation*}
where $'$ denotes the partial derivative concerning $d_{1}$. Hence, we get
\begin{equation}\label{eq414}
Re\left(\frac{\partial\lambda}{\partial d_{1}}|_{\lambda_{1,2}=\pm\sqrt{P_{2}}i}\right)=\frac{P_{1}'P_{2}}{P_{2}+P_{1}^{2}}|_{d_{1}=\tilde{d}_{1}}.
\end{equation}
From the condition $P_{1}'P_{2}\neq0$, then we have $Re(\frac{\partial\lambda}{\partial d_{1}})|_{d_{1}=\tilde{d}_{1}}\neq0$, which confirms that the transversality condition for the Hopf bifurcation is satisfied. Therefore, a Hopf bifurcation will occur at the interior equilibrium $\bar E$ of system (2) near the critical value $d_{1}=\tilde{d}_{1}$.
\end{proof}

\appendix{Stability and Direction of Hopf bifurcation}
\begin{proof}
We first perform a translation transformation on system (2), moving the boundary equilibrium
$E_{1}(d_{1}^{0})=(\frac{\alpha_{1}-1}{2\alpha_{1}},\frac{(\alpha_{1}+1)^{2}}{4\alpha_{1}},0)$ to the origin. i.e.,
\begin{equation}
\left\{
\begin{array}{l}
x_{1}=x-\frac{\alpha_{1}-1}{2\alpha_{1}}\\
x_{2}=y-\frac{(\alpha_{1}+1)^{2}}{4\alpha_{1}}\\
x_{3}=z.\\
\end{array}
\right.
\end{equation}
Then, system (2) becomes system (\ref{eq8}) as follow:
\begin{equation}\label{eq8}
    \dot{X}=J_{E_{1}}(d_{1}^{0})X+F(X),X=(x_{1},x_{2},x_{3})^{T},F(X)=(F_{1},F_{2},F_{3})^{T},
\end{equation}
where $F_{i}(X)(i=1,2,3)$ are the non-linear terms of the right hand functions of (\ref{eq8}), detailed as follows:
\begin{equation*}
\begin{aligned}
F_{1}&=-\frac{x_{1}^{2} \left(\alpha_{1}-1\right)}{\alpha_{1}+1}-\frac{4 x_{1} x_{2}}{\left(\alpha_{1}+1\right)^{2}}-\frac{4\alpha_{1}^{2} x_{1} x_{3} }{\left(\left(\alpha_{2}+2\right) \alpha_{1}-\alpha_{2}\right)^{2}}
-\frac{4\alpha_{1} x_{1}^{3} }{\left(\alpha_{1}+1\right)^{2}}
+\frac{8\alpha_{1} x_{1}^{2} x_{2}}
{\left(\alpha_{1}+1\right)^{3}}
\\&~~~
+\frac{8 \alpha_{2} \alpha_{1}^{3}x_{1}^{2} x_{3} }{\left(\left(\alpha_{2}+2\right) \alpha_{1}-\alpha_{2}\right)^{3}}+O\left(X^{4}\right),\\
F_{2}&=-\frac{2 \gamma_{1} x_{1}^{2}}{\alpha_{1}+1}+\frac{4 \gamma_{1} x_{1} x_{2}}{\left(\alpha_{1}+1\right)^{2}}-\frac{16 \gamma_{1}  e_{1} \alpha_{1}^{2}x_{2} x_{3}}{\left(\alpha_{1}^{2} h +\left(2 h +4\right) \alpha_{1}+h \right)^{2}}
+\frac{4 \alpha_{1} \gamma_{1} x_{1}^{3}}{\left(\alpha_{1}+1\right)^{2}}-\frac{8 \gamma_{1} \alpha_{1} x_{1}^{2} x_{2}}{\left(\alpha_{1}+1\right)^{3}}
\\&
~~~+\frac{64 \gamma_{1}e_{1} h \alpha_{1}^{3} x_{2}^{2} x_{3}}{\left(\alpha_{1}^{2} h +\left(2 h +4\right) \alpha_{1}+h \right)^{3}}+O\left(X^{4}\right),\\
F_{3}&=\frac{4 \gamma_{2} \alpha_{1}^{2} x_{1} x_{3}}{\left(\left(\alpha_{2}+2\right) \alpha_{1}-\alpha_{2}\right)^{2}}+\frac{16 e_{2} \gamma_{2} \alpha_{1}^{2}x_{2} x_{3} }{\left(\alpha_{1}^{2} h +\left(2 h +4\right) \alpha_{1}+h \right)^{2}}
-\frac{8 \gamma_{2} \alpha_{2} \alpha_{1}^{3}x_{1}^{2} x_{3} }{\left(\left(\alpha_{2}+2\right) \alpha_{1}-\alpha_{2}\right)^{3}}
\\&~~~
-\frac{64 e_{2} \gamma_{2}\alpha_{1}^{3} h x_{2}^{2} x_{3} }{\left(\alpha_{1}^{2} h +\left(2 h +4\right) \alpha_{1}+h \right)^{3}}+O \left(X^{4}\right).
\end{aligned}
\end{equation*}
Secondly, introduce an invertible transformation as follows:
\begin{equation*}
(x_{1},x_{2},x_{3})^{T}=P(y_{1},y_{2},y_{3})^{T},
 \end{equation*}
where
\begin{equation*}
P=\left(
  \begin{array}{ccc}
    \xi_{1} & -\frac{\sqrt{\alpha_{1}^2-1}}{\sqrt{\gamma_{1}}(1+\alpha_{1})}
    &\frac{\sqrt{\alpha_{1}^2-1}}{\sqrt{\gamma_{1}}(1+\alpha_{1})}  \\
    \xi_{2}& 1 & 1 \\
    1 & 0 & 0 \\
  \end{array}
\right),
        \end{equation*}
\begin{equation*}
\begin{aligned}
\xi_{1}&=\{\left(\alpha_{1}-1\right) \left(\alpha_{1}+1\right) \left(h \alpha_{1}^{2}+2 h \alpha_{1}+h +4 \alpha_{1}\right) [e_{1} \left(\alpha_{1}+1\right) \left(\alpha_{1} \alpha_{2}+2 \alpha_{1}-\alpha_{2}\right)^{2} (h \alpha_{1}^{2}+2 h \alpha_{1}
\\&
+h +4 \alpha_{1}) \gamma_{1}+4 \alpha_{1}^{2} \gamma_{2} \left(h +1\right)(h \alpha_{1}^{3} \alpha_{2} d_{2}+2 h \alpha_{1}^{3} d_{2}+h \alpha_{1}^{2} \alpha_{2} d_{2}-\alpha_{1}^{3} \alpha_{2} e_{2}-h \alpha_{1}^{3}+4 h \alpha_{1}^{2} d_{2}
\\&
-h \alpha_{1} \alpha_{2} d_{2}-2 \alpha_{1}^{3} e_{2}+4 \alpha_{1}^{2} \alpha_{2} d_{2}-\alpha_{1}^{2} \alpha_{2} e_{2}-h \alpha_{1}^{2}+2 h \alpha_{1} d_{2}-h \alpha_{2} d_{2}+8 \alpha_{1}^{2} d_{2}-4 \alpha_{1}^{2} e_{2}
\\&
-4 \alpha_{1} \alpha_{2} d_{2}+\alpha_{1} \alpha_{2} e_{2}+h \alpha_{1}-4 \alpha_{1}^{2}-2 \alpha_{1} e_{2}
+\alpha_{2} e_{2}+h +4 \alpha_{1})]\}/\{4 \left(h +1\right)[\left(\alpha_{1}-1\right)
\\&
 \left(h \alpha_{1}^{2}+2 h \alpha_{1}+h +4 \alpha_{1}\right)^{2}
 \left(\alpha_{1} \alpha_{2}+2 \alpha_{1}-\alpha_{2}\right)^{2}\gamma_{1}
 +\alpha_{1}^{2} \gamma_{2}^{2} \left(\alpha_{1}+1\right)\left(h \alpha_{1}^{3} \alpha_{2} d_{2}+2 h \alpha_{1}^{3} d_{2}-h \alpha_{1}^{2}
\right.\\
&\left.+h \alpha_{1}^{2} \alpha_{2} d_{2}-\alpha_{1}^{3} \alpha_{2} e_{2}-h \alpha_{1}^{3}+4 h \alpha_{1}^{2} d_{2}-h \alpha_{1} \alpha_{2} d_{2}-2 \alpha_{1}^{3} e_{2}+4 \alpha_{1}^{2} \alpha_{2} d_{2}-\alpha_{1}^{2} \alpha_{2} e_{2}+2 h \alpha_{1} d_{2}
\right.\\
&\left.-h \alpha_{2} d_{2}+8 \alpha_{1}^{2} d_{2}-4 \alpha_{1}^{2} e_{2}-4 \alpha_{1} \alpha_{2} d_{2}+\alpha_{1} \alpha_{2} e_{2}+h \alpha_{1}-4 \alpha_{1}^{2}-2 \alpha_{1} e_{2}+\alpha_{2} e_{2}+h +4 \alpha_{1}\right)^{2}]\},\\
\xi_{2}&=\{\alpha_{1} \gamma_{1} \left(\alpha_{1}+1\right) \left(\alpha_{1} \alpha_{2}+2 \alpha_{1}-\alpha_{2}\right) \left(h \alpha_{1}^{2}+2 h \alpha_{1}+h +4 \alpha_{1}\right) [h \alpha_{1}^{5} \alpha_{2} d_{2} e_{1} \gamma_{2}+2 h \alpha_{1}^{5} d_{2} e_{1} \gamma_{2}
\\&
+3 h \alpha_{1}^{4} \alpha_{2} d_{2} e_{1} \gamma_{2}-\alpha_{1}^{5} \alpha_{2} e_{1} e_{2} \gamma_{2}-h \alpha_{1}^{5} e_{1} \gamma_{2}+8 h \alpha_{1}^{4} d_{2} e_{1} \gamma_{2}+2 h \alpha_{1}^{3} \alpha_{2} d_{2} e_{1} \gamma_{2}-2 \alpha_{1}^{5} e_{1} e_{2} \gamma_{2}
\\&
+4 \alpha_{1}^{4} \alpha_{2} d_{2} e_{1} \gamma_{2}-3 \alpha_{1}^{4} \alpha_{2} e_{1} e_{2} \gamma_{2}-3 h \alpha_{1}^{4} e_{1} \gamma_{2}+12 h \alpha_{1}^{3} d_{2} e_{1} \gamma_{2}-2 h \alpha_{1}^{2} \alpha_{2} d_{2} e_{1} \gamma_{2}+8 \alpha_{1}^{4} d_{2} e_{1} \gamma_{2}
\\&
-8 \alpha_{1}^{4} e_{1} e_{2} \gamma_{2}+4 \alpha_{1}^{3} \alpha_{2} d_{2} e_{1} \gamma_{2}
-2 \alpha_{1}^{3} \alpha_{2} e_{1} e_{2} \gamma_{2}-2 h \alpha_{1}^{3} e_{1} \gamma_{2}+8 h \alpha_{1}^{2} d_{2} e_{1} \gamma_{2}-3 h \alpha_{1} \alpha_{2} d_{2} e_{1} \gamma_{2}
\\&
-4 \alpha_{1}^{4} e_{1} \gamma_{2}+16 \alpha_{1}^{3} d_{2} e_{1} \gamma_{2}-12 \alpha_{1}^{3} e_{1} e_{2} \gamma_{2}-4 \alpha_{1}^{2} \alpha_{2} d_{2} e_{1} \gamma_{2}+2 \alpha_{1}^{2} \alpha_{2} e_{1} e_{2} \gamma_{2}-4 h^{2} \alpha_{1}^{3}+2 h \alpha_{1}^{2} e_{1} \gamma_{2}
\\&
+2 h \alpha_{1} d_{2} e_{1} \gamma_{2}-h \alpha_{2} d_{2} e_{1} \gamma_{2}-4 \alpha_{1}^{3} e_{1} \gamma_{2}+8 \alpha_{1}^{2} d_{2} e_{1} \gamma_{2}-8 \alpha_{1}^{2} e_{1} e_{2} \gamma_{2}-4 \alpha_{1} \alpha_{2} d_{2} e_{1} \gamma_{2}+3 \alpha_{1} \alpha_{2} e_{1} e_{2} \gamma_{2}
\\&
-4 h^{2} \alpha_{1}^{2}-4 h \alpha_{1}^{3}+3 h \alpha_{1} e_{1} \gamma_{2}+4 \alpha_{1}^{2} e_{1} \gamma_{2}
-2 \alpha_{1} e_{1} e_{2} \gamma_{2}+\alpha_{2} e_{1} e_{2} \gamma_{2}+4 h^{2} \alpha_{1}-20 h \alpha_{1}^{2}+h e_{1} \gamma_{2}
\\&
+4 \alpha_{1} e_{1} \gamma_{2}+4 h^{2}+20 h \alpha_{1}-16 \alpha_{1}^{2}+4 h +16 \alpha_{1}]\}/\{4 \left(h +1\right)[(\alpha_{1}-1)\left(h \alpha_{1}^{2}+2 h \alpha_{1}+h +4 \alpha_{1}\right)^{2}
\\&
(\alpha_{1} \alpha_{2}+2 \alpha_{1}-\alpha_{2})^{2} \gamma_{1}+\alpha_{1}^{2} \gamma_{2}^{2} \left(\alpha_{1}+1\right)(h \alpha_{1}^{3} \alpha_{2} d_{2}+2 h \alpha_{1}^{3} d_{2}+h \alpha_{1}^{2} \alpha_{2} d_{2}-\alpha_{1}^{3} \alpha_{2} e_{2}-h \alpha_{1}^{3}+4 h \alpha_{1}^{2} d_{2}\\
&
-h \alpha_{1} \alpha_{2} d_{2}-2 \alpha_{1}^{3} e_{2}+4 \alpha_{1}^{2} \alpha_{2} d_{2}-\alpha_{1}^{2} \alpha_{2} e_{2}-h \alpha_{1}^{2}+2 h \alpha_{1} d_{2}-h \alpha_{2} d_{2}+8 \alpha_{1}^{2} d_{2}-4 \alpha_{1}^{2} e_{2}-4 \alpha_{1} \alpha_{2} d_{2}
\\&
+\alpha_{1} \alpha_{2} e_{2}+h \alpha_{1}-4 \alpha_{1}^{2}-2 \alpha_{1} e_{2}+\alpha_{2} e_{2}+h +4 \alpha_{1})^{2}]\}.
\end{aligned}
\end{equation*}
Then, system (\ref{eq8}) becomes
\begin{equation}\label{eq9}
    \dot{Y}=P^{-1}J_{E_{1}}(d_{1}^{0})PY+Q, Y=(y_{1},y_{2},y_{3})^{T}, Q=(Q^{1},Q^{2},Q^{3})^{T},
\end{equation}
where
\begin{equation*}
P^{-1}J_{E_{1}}(d_{1}^{0})P=\left(
                   \begin{array}{ccc}
                     \beta &
                    0  &
                     0\\
                    0 &
                      -\frac{\sqrt{\gamma_{1}(\alpha_{1}^{2}-1)}}{\alpha_{1} \left(\alpha_{1}+1\right)}  & 0 \\
                     0 & 0 & \frac{\sqrt{\gamma_{1}(\alpha_{1}^{2}-1)}}{\alpha_{1} \left(\alpha_{1}+1\right)}\\
                   \end{array}
                 \right),\\
\end{equation*}
{\footnotesize
\begin{equation*}
 \begin{aligned}
 &\beta=\gamma_{2} \left(\frac{\alpha_{1}-1}{\left(\alpha_{2}+2\right) \alpha_{1}-\alpha_{2}}+\frac{e_{2} \left(\alpha_{1}+1\right)^{2}}{h \alpha_{1}^{2}+\left(2 h +4\right) \alpha_{1}+h}-d_{2}\right),\\
&Q^{1}=\frac{4 \alpha_{1}^{2} \gamma_{2} y_{1} y_{3}}{\left(\alpha_{2}-\left(\alpha_{2}+2\right) \alpha_{1}\right)^{2}}+\frac{16 e_{2} \gamma_{2} \alpha_{1}^{2}y_{2} y_{3}}{\left(\alpha_{1}^{2} h +\left(2 h +4\right) \alpha_{1}+h \right)^{2}}
                              +\frac{8 \alpha_{1}^{3} \alpha_{2} \gamma_{2} y_{1}^{2} y_{3}}{\left(\alpha_{2}
                              -\left(\alpha_{2}+2\right) \alpha_{1}\right)^{3}}\\
 &-\frac{64 e_{2} \gamma_{2} \alpha_{1}^{3} h y_{2}^{2} y_{3} }{\left(\alpha_{1}^{2} h +\left(2 h +4\right) \alpha_{1}+h \right)^{3}}+O\left(Y^{4}\right),\\
&Q^{2}=-\frac{\gamma_{1} y_{1}^{2}}{\alpha_{1}+1}+\frac{2 \gamma_{1} y_{1} y_{2}}{\left(\alpha_{1}+1\right)^{2}}-\frac{8 \gamma_{1} e_{1} \alpha_{1}^{2} y_{2} y_{3}}{\left(\alpha_{1}^{2} h +\left(2 h +4\right) \alpha_{1}+h \right)^{2}}
                              -\frac{4 \gamma_{1}\alpha_{1} y_{1}^{2} y_{2} }{\left(\alpha_{1}+1\right)^{3}}\\
                                \end{aligned}
\end{equation*}
                              \begin{equation*}
 \begin{aligned}
                              &+\frac{2 \alpha_{1} \gamma_{1} y_{1}^{3}}{\left(\alpha_{1}+1\right)^{2}}+\frac{1}{\sqrt{\alpha_{1}^{2}-1}}\left\{(\frac{\xi_{1} \sqrt{\gamma_{1}}}{2}-\frac{\xi_{2} \sqrt{\alpha_{1}^{2}-1}}{2}+\frac{\alpha_{1} \xi_{1} \sqrt{\gamma_{1}}}{2})\left[\frac{4\alpha_{1}^{2} \gamma_{2} y_{1} y_{3}}{\left(\alpha_{2}-\left(\alpha_{2}+2\right) \alpha_{1}\right)^{2}}\right.\right.\\
                              &\left.\left.+\frac{8 \alpha_{1}^{3} \alpha_{2} \gamma_{2} y_{1}^{2} y_{3}}{\left(\alpha_{2}-\left(\alpha_{2}+2\right) \alpha_{1}\right)^{3}}
                              +\frac{16 e_{2} \gamma_{2}\alpha_{1}^{2} y_{2} y_{3} }{\left(\alpha_{1}^{2} h +\left(2 h +4\right) \alpha_{1}+h \right)^{2}}
                              -\frac{64 e_{2} \gamma_{2}  \alpha_{1}^{3} hy_{2}^{2} y_{3} }{\left(\alpha_{1}^{2} h +\left(2 h +4\right) \alpha_{1}+h \right)^{3}}\right]\right\}\\
                              &+\frac{1}{2\sqrt{\alpha_{1}^{2}-1}}\left\{\sqrt{\gamma_{1}}\left(\alpha_{1}+1\right)
                              \left[\frac{(\alpha_{1}-1)y_{1}^{2} }{\alpha_{1}+1}
                              +\frac{4 y_{1} y_{2}}{\left(\alpha_{1}+1\right)^{2}}+\frac{4  \alpha_{1}y_{1}^{3} }{\left(\alpha_{1}+1\right)^{2}}-\frac{8\alpha_{1} y_{1}^{2} y_{2}}{\left(\alpha_{1}+1\right)^{3}}\right.\right.\\
                              &\left.\left.+\frac{4 \alpha_{1}^{2} y_{1} y_{3}}{\left(\alpha_{2}-\left(\alpha_{2}+2\right) \alpha_{1}\right)^{2}}+\frac{8 \alpha_{1}^{3} \alpha_{2} y_{1}^{2} y_{3}}{\left(\alpha_{2}-\left(\alpha_{2}+2\right) \alpha_{1}\right)^{3}}\right]\right\}+\frac{32 \gamma_{1}e_{1} h \alpha_{1}^{3} y_{2}^{2} y_{3} }{\left(\alpha_{1}^{2} h +\left(2 h +4\right) \alpha_{1}+h \right)^{3}}+O\left(Y^{4}\right),\\
Q^{3}&=-\frac{1}{\sqrt{\alpha_{1}^{2}-1}}\left\{(\frac{\xi_{1} \sqrt{\gamma_{1}}}{2}+\frac{\xi_{2} \sqrt{\alpha_{1}^{2}-1}}{2}+\frac{\alpha_{1} \xi_{1} \sqrt{\gamma_{1}}}{2})\left[\frac{4 \alpha_{1}^{2} \gamma_{2} y_{1} y_{3}}{\left(\alpha_{2}-\left(\alpha_{2}+2\right)\alpha_{1}\right)^{2}}\right.\right.\\
                              &\left.\left.+\frac{16 e_{2} \gamma_{2}\alpha_{1}^{2} y_{2} y_{3} }{\left(\alpha_{1}^{2} h +\left(2 h +4\right) \alpha_{1}+h \right)^{2}}
                              +\frac{8 \alpha_{1}^{3} \alpha_{2} \gamma_{2} y_{1}^{2} y_{3}}{\left(\alpha_{2}-\left(\alpha_{2}+2\right) \alpha_{1}\right)^{3}}-\frac{64 e_{2} \gamma_{2} \alpha_{1}^{3} h y_{2}^{2} y_{3}}{\left(\alpha_{1}^{2} h +\left(2 h +4\right) \alpha_{1}+h \right)^{3}}\right]\right\}\\
                              &-\frac{1}{2\sqrt{\alpha_{1}^{2}-1}}\left\{\sqrt{\gamma_{1}}\left(\alpha_{1}+1\right) \left[\frac{(\alpha_{1}-1)y_{1}^{2}}{\alpha_{1}+1}+\frac{4y_{1} y_{2}}{\left(\alpha_{1}+1\right)^{2}}+\frac{4\alpha_{1} y_{1}^{3}}{\left(\alpha_{1}+1\right)^{2}}-\frac{8\alpha_{1} y_{1}^{2} y_{2}}{\left(\alpha_{1}+1\right)^{3}}\right.\right.\\
                              &\left.\left.+\frac{4 \alpha_{1}^{2} y_{1} y_{3}}{\left(\alpha_{2}-\left(\alpha_{2}+2\right) \alpha_{1}\right)^{2}}
                              +\frac{8 \alpha_{1}^{3} \alpha_{2} y_{1}^{2} y_{3}}{\left(\alpha_{2}-\left(\alpha_{2}+2\right) \alpha_{1}\right)^{3}}\right]\right\}-\frac{\gamma_{1} y_{1}^{2}}{\alpha_{1}+1}+\frac{2 \alpha_{1} \gamma_{1} y_{1}^{3}}{\left(\alpha_{1}+1\right)^{2}}+\frac{2 \gamma_{1} y_{1} y_{2}}{\left(\alpha_{1}+1\right)^{2}}\\
                              &-\frac{4 \gamma_{1} y_{1}^{2} y_{2} \alpha_{1}}{\left(\alpha_{1}+1\right)^{3}}-\frac{8 \gamma_{1} y_{2} y_{3} e_{1} \alpha_{1}^{2}}{\left(\alpha_{1}^{2} h +\left(2 h +4\right) \alpha_{1}+h \right)^{2}}+\frac{32 \gamma_{1} y_{2}^{2} y_{3} e_{1} h \alpha_{1}^{3}}{\left(\alpha_{1}^{2} h +\left(2 h +4\right) \alpha_{1}+h \right)^{3}}+O\left(Y^{4}\right).
                              \end{aligned}
\end{equation*}}
Next, we calculate the following of the nonlinear terms
\begin{align*}
g_{20}(0,0,0)&=\frac{\sqrt{\gamma_{1} \left(\alpha_{1}^{2}-1\right)}}{\left(\alpha_{1}+1\right) \left(\alpha_{1}^{2}-1\right)}+\frac{2\gamma_{1}-i\gamma_{1} \left(\alpha_{1}+3\right) }{2\left(\alpha_{1}+1\right)^{2}}+\frac{i\left[\left(\alpha_{1}^{2}-5\right) \sqrt{\gamma_{1}}\sqrt{\alpha_{1}^{2}-1}\left(\alpha_{1}-1\right)\right]}{4 \left(\alpha_{1}+1\right)^{2} \left(\alpha_{1}-1\right)},\\
g_{11}(0,0,0)&=\frac{i}{4} \left(-\frac{2 \gamma_{1}}{\alpha_{1}+1}+\frac{\sqrt{\gamma_{1} \left(\alpha_{1}^{2}-1\right)}}{\alpha_{1}+1}\right),~~W_{11}(0,0,0)=\frac{2 \gamma_{1}+\sqrt{\gamma_{1} \left(\alpha_{1}^{2}-1\right)}}{4 \beta_{1} \left(\alpha_{1}+1\right)},\\
g_{02}(0,0,0)&=-\frac{\sqrt{\gamma_{1} \left(\alpha_{1}^{2}-1\right)}}{\left(\alpha_{1}+1\right) \left(\alpha_{1}^{2}-1\right)}-\frac{\gamma_{1}}{\left(\alpha_{1}+1\right)^{2}}+\frac{i \left(\left(\alpha_{1}^{2}+3\right) \sqrt{\gamma_{1}}\sqrt{\alpha_{1}^{2}-1}-2 \gamma_{1} \left(\alpha_{1}-1\right)^{2}\right)}{4 \left(\alpha_{1}+1\right)^{2} \left(\alpha_{1}-1\right)},\\
g_{21}(0,0,0)&=G_{101} W_{20}+2 G_{110} W_{11}+G_{21},~~G_{21}=\frac{i}{8},\\
G_{110}(0,0,0)&=\frac{2 \alpha_{1}^{2} \gamma_{2}}{\left(\alpha_{2}-\left(\alpha_{2}+2\right) \alpha_{1}\right)^{2}}+\frac{4 \left(-\xi_{2} \sqrt{\alpha_{1}^{2}-1}+\xi_{1} \sqrt{\gamma_{1}} \left(\alpha_{1}+1\right)\right) e_{2} \gamma_{2} \alpha_{1}^{2}}{\left(\alpha_{1}^{2} h +\left(2 h +4\right) \alpha_{1}+h \right)^{2} \sqrt{\alpha_{1}^{2}-1}},\\
&-\frac{4 \gamma_{1} e_{1} \alpha_{1}^{2}}{\left(\alpha_{1}^{2} h +\left(2 h +4\right) \alpha_{1}+h \right)^{2}}
+i\left[\frac{(\xi_{1} \sqrt{\gamma_{1}}\left(\alpha_{1}-1\right)-\xi_{2} \sqrt{\alpha_{1}^{2}-1}) \alpha_{1}^{2} \gamma_{2}}{\left(\alpha_{2}-\left(\alpha_{2}+2\right) \alpha_{1}\right)^{2} \sqrt{\alpha_{1}^{2}-1}}\right.\\
&\left.+\frac{\sqrt{\gamma_{1}} \left(\alpha_{1}+1\right) \alpha_{1}^{2}}{\left(\alpha_{2}-\left(\alpha_{2}+2\right) \alpha_{1}\right)^{2} \sqrt{\alpha_{1}^{2}-1}}-\frac{8 e_{2} \gamma_{2} \alpha_{1}^{2}}{\left(\alpha_{1}^{2} h +\left(2 h +4\right) \alpha_{1}+h \right)^{2}}\right],\\
W_{20}(0,0,0)&=-\frac{(2i \sqrt{\gamma_{1}}\sqrt{\alpha_{1}^{2}-1}+\beta\alpha_{1}^{2}+\beta \alpha_{1}) \alpha_{1} }{\{\left(\alpha_{1}+1\right)
\left(\beta^{2}\alpha_{1}^{3}+\beta^{2}\alpha_{1}^{2}+4 \alpha_{1} \gamma_{1}-4 \gamma_{1}\right)\}}
\left\{\left[-\sqrt{\gamma_{1} \left(\alpha_{1}^{2}-1\right)}\alpha_{1}^{2}\right.\right.\\
&\left.\left.+4i\alpha_{1} \gamma_{1}-2 \alpha_{1}^{2} \gamma_{1}
+4 i \sqrt{\gamma_{1}
\left(\alpha_{1}^{2}-1\right)}
-4i \gamma_{1}+\sqrt{\gamma_{1} \left(\alpha_{1}^{2}-1\right)}+2 \gamma_{1}\right]\right\},\\
\end{align*}
\begin{equation*}
\begin{aligned}
G_{101}(0,0,0)&=\frac{2\alpha_{1}^{2} \gamma_{2}}{\left(\alpha_{2}-\left(\alpha_{2}+2\right) \alpha_{1}\right)^{2}}-\frac{4 \left(-\xi_{2} \sqrt{\alpha_{1}^{2}-1}+\xi_{1} \sqrt{\gamma_{1}} \left(\alpha_{1}+1\right)\right) e_{2} \gamma_{2} \alpha_{1}^{2}}{\left(\alpha_{1}^{2} h +\left(2 h +4\right) \alpha_{1}+h \right)^{2} \sqrt{\alpha_{1}^{2}-1}}\\
&+\frac{4 \gamma_{1} e_{1} \alpha_{1}^{2}}{\left(\alpha_{1}^{2} h +\left(2 h +4\right) \alpha_{1}+h \right)^{2}}+i\left[-\frac{\left(-\xi_{2} \sqrt{\alpha_{1}^{2}-1}+\xi_{1} \sqrt{\gamma_{1}}\left(\alpha_{1}+1\right)\right) \alpha_{1}^{2} \gamma_{2}}{\left(\alpha_{2}-\left(\alpha_{2}+2\right) \alpha_{1}\right)^{2} \sqrt{\alpha_{1}^{2}-1}}\right.\\
&\left.-\frac{\sqrt{\gamma_{1}} \left(\alpha_{1}+1\right) \alpha_{1}^{2}}{\left(\alpha_{2}-\left(\alpha_{2}+2\right) \alpha_{1}\right)^{2} \sqrt{\alpha_{1}^{2}-1}}+\frac{8 e_{2} \gamma_{2} \alpha_{1}^{2}}{\left(\alpha_{1}^{2} h +\left(2 h +4\right) \alpha_{1}+h \right)^{2}}\right].
\end{aligned}
               \end{equation*}
Finally, according to [Singh et al. 2013], we have
\begin{equation*}\label{eq481}
\begin{aligned}
l_{1}&=\frac{\alpha_{1}(\alpha_{1}+1)i}{2\sqrt{\gamma_{1}(\alpha_{1}^2-1)}}\left(g_{20}g_{11}-2|g_{11}|^{2}-\frac{1}{3}|g_{02}|^{2}\right)+\frac{g_{21}}{2},~l_{1}^{0}=Re\{l_{1}\},~\sigma=-\frac{4 l_{1}^{0}}{1-\alpha_{1}^2},\end{aligned}
\end{equation*}
where
\begin{equation*}
\begin{aligned}
 &l_{1}^{0}=\{64 \alpha_{1}^{4} [(h^{2} \alpha_{1}^{4} \gamma_{2}+4 \gamma_{2} h \left(h +2\right) \alpha_{1}^{3}+(\left(2 \alpha_{2}^{2} e_{2} \xi_{2}+8 e_{2} \alpha_{2}\xi_{2}+6 h^{2}+8 e_{2} \xi_{2}+16 h +16\right) \gamma_{2}\\
&+2 e_{1} \gamma_{1} \left(\alpha_{2}+2\right)^{2}) \alpha_{1}^{2}
+\left(\left(-4 \alpha_{2}^{2} e_{2} \xi_{2}-8 e_{2} \alpha_{2} \xi_{2}+4 h^{2}+8 h \right) \gamma_{2}-4 e_{1} \alpha_{2} \gamma_{1} \left(\alpha_{2}+2\right)\right) \alpha_{1}\\
&+\left(2 \alpha_{2}^{2} e_{2} \xi_{2}+h^{2}\right) \gamma_{2}+2 \alpha_{2}^{2} e_{1} \gamma_{1})(\alpha_{1}^{2}-1)^{1/2}
-2 \xi_{1} e_{2} \gamma_{2} \left(\left(\alpha_{2}+2\right) \alpha_{1}-\alpha_{2}\right)^{2} \left(\alpha_{1}+1\right)\gamma_{1}^{1/2}]\\
&\{\{\beta\left(\alpha_{1}+1\right)^{2}\gamma_{1}^{1/2}+32 \gamma_{1}^{3} \left(\alpha_{1}-1\right)) \left(\alpha_{1}-1\right) (\alpha_{1}^{2}-1)^{1/2}\}/64+(\gamma_{1}^{5/2}
+ \beta\gamma_{1}/32)\alpha_{1}^{3}\\
&+(-\gamma_{1}^{5/2}+\beta\gamma_{1}/32)\alpha_{1}^{2}
+(-\gamma_{1}^{5/2}-\beta\gamma_{1}/32)\alpha_{1}
+\gamma_{1}^{5/2}+\left(\alpha_{1}^{2}-1\right)^{3/2}\gamma_{1}^{2}/2
-\beta \gamma_{1}/32\}\}\\
&/\{\left(\alpha_{1}+1\right)(\alpha_{1}^{2}-1)^{1/2}(\alpha_{1}^{3} \beta^{2}+\alpha_{1}^{2} \beta^{2}
+4 \gamma_{1} (\alpha_{1}-1))(\alpha_{1} \alpha_{2}+2 \alpha_{1}-\alpha_{2})^{2}(\alpha_{1}^{2} h +2 h \alpha_{1}+h\\
 &+4 \alpha_{1})^{2}\}-\{[12 \gamma_{1} \left(\alpha_{1}^{2}-2 \alpha_{1}-3\right) (\alpha_{1}^{2}-1)\gamma_{1}^{1/2}+12 \left(\alpha_{1}-2\gamma_{1}+1\right) \gamma_{1} \left(\alpha_{1}-1\right)\left(\alpha_{1}+1\right)] \alpha_{1}\}\\
  &/\{384 \left(\alpha_{1}+1\right)^{3} \left(\alpha_{1}-1\right)
[\gamma_{1}(\alpha_{1}^{2}-1)]^{1/2}\}+\{2 \gamma_{1}+[\gamma_{1}(\alpha_{1}^{2}-1)]^{1/2}\}
/[{4 \beta \left(\alpha_{1}+1\right)}]\\
&\{2 \alpha_{1}^{2} \gamma_{2}/[\left(\alpha_{2}-\left(\alpha_{2}+2\right) \alpha_{1}\right)^{2}]+[4(-\xi_{2}(\alpha_{1}^{2}-1)^{1/2}+\xi_{1} \gamma_{1}^{1/2}\left(\alpha_{1}+1\right)) e_{2}
\gamma_{2}\alpha_{1}^{2}]\\
&/[(\alpha_{1}^{2} h+\left(2 h +4\right) \alpha_{1}+h)^{2} (\alpha_{1}^{2}-1)^{1/2}]
-4 \gamma_{1} e_{1} \alpha_{1}^{2}/[\left(\alpha_{1}^{2} h +\left(2 h +4\right) \alpha_{1}+h \right)^{2}]\}.
\end{aligned}
\end{equation*}
Thus, according to [Singh et al. 2013], the limit cycle of system is unstable and the hopf bifurcation is subcritical if $\sigma<0$, and the limit cycle of system is stable and the hopf bifurcation is supercritical if $\sigma>0$. For the equilibrium $E_{2},\bar E$ can be derived similarly and will not be proved here.
\end{proof}


\end{document}
